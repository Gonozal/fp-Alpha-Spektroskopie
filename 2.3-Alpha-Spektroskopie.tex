% Klassifiziert den Dokumenten-Typ
% Doku: http://exp1.fkp.physik.tu-darmstadt.de/tuddesign/
% Farben: http://www.tu-darmstadt.de/media/medien_stabsstelle_km/services/medien_cd/das_bild_der_tu_darmstadt.pdf
%  bigchapter: Chapter haben doppelte Schriftgröße
%  linedtoc: Linien im Inhaltsverzeichnis wie bei Überschriften
%  colorbacktitle: Der Dokumenten-Titel wird mir der Accentfarbe hinterlegt
\documentclass[bigchapter,colorback,accentcolor=tud4b,linedtoc,11pt]{tudreport}

% Input Dokument hat das Encoding UTF-8
\usepackage[utf8]{inputenc}
% Wichtiges Paket für Links und verlinktes Inhaltsverzeichnis
\usepackage{hhline}
\usepackage[ngerman]{hyperref}
% Paket für Fußnoten
\usepackage[stable]{footmisc}
% Paket für amsmath (aligned mathe formeln)
\usepackage{amsmath}
% Paket für Bibliotheks-Verzeichnis, square: Verwende eckige statt runde klammern
% \usepackage[square]{natbib}
% Paket zum Plotten von Datensätzen
\usepackage{pgfplots}
\usepgfplotslibrary{patchplots}


\pgfkeys{%
  /pgfplots/default/.style={%
    /pgf/number format/use comma,
    legend pos=north east,
    width=0.9\linewidth,
    height=0.7\linewidth,
    scale only axis,
    xmin=0,
    ymin=0,
    grid=both,
    tick align=outside,
    tickpos=left,
    minor x tick num=3,
    minor y tick num=4,
    minor grid style={dotted,thin},
    x tick label style={/pgf/number format/.cd,%
      set thousands separator={},
      set decimal separator={,}
    },%
    y tick label style={/pgf/number format/.cd,%
      set thousands separator={},
      set decimal separator={,}
    },%
  }
}

% Anhänge für Original-Messdaten
\usepackage{fancyvrb}

% redefine \VerbatimInput
\RecustomVerbatimCommand{\VerbatimInput}{VerbatimInput}%
{fontsize=\footnotesize,
 %
 frame=lines,  % top and bottom rule only
 framesep=2em, % separation between frame and text
 fontsize=\scriptsize,
 %
 labelposition=topline,
 %
 commandchars=\|\(\), % escape character and argument delimiters for
                      % commands within the verbatim
 commentchar=*        % comment character
}

% Polar Plots
\usetikzlibrary{pgfplots.polar}
% Verwende deutsche Bezeichner für Inhaltsverzeichnis, ... (ngerman = New German: neue Rechtschreibung)
\usepackage{ngerman}
% Deutsche Zahlen (entfernt z.B. das Leerzeichen nach einem Dezimal-Komma)
\usepackage{ziffer} 

\usepackage[verbose]{placeins}

%wegen Grafikverschiebung hinzugefügt
\usepackage{float}

%\usepackage{graphicx}
%\usepackage{caption}
\usepackage{subcaption} %Für subfigures

% PDF-Optionen
\hypersetup{%
  pdftitle={TU Darmstadt \- Physikalisches Praktikum für Fortgeschrittene},
  pdfauthor={Esra Bauer und Sören Link},
  pdfsubject={Versuch 2.3},
  pdfview=FitH,
}
% Nummeriere formeln in Subsections einzeln
% Kleines makro zur assymetrischen Fehlerangabe

% Entspricht-Zeichen
\usepackage{scalerel}

\newcommand\equalhat{%
\let\savearraystretch\arraystretch
\renewcommand\arraystretch{0.3}
\begin{array}{c}
\stretchto{
    \scalerel*[\widthof{=}]{\wedge}
    {\rule{1ex}{3ex}}%
}{0.5ex}\\ 
=%
\end{array}
\let\arraystretch\savearraystretch
}
%BEGINN TITELSEITE

\title{Tieftemperaturmessung an suprafluidem Helium}

\subtitle{Esra Bauer \\Sören Link}

\subsubtitle{Betreuer: Simon Frydrych \hfill Versuchsdatum: 15. Juni 2015}

\author{Esra Bauer, Sören Link}

%\settitlepicture{img/title.jpg}

\institution{Physikalisches Praktikum \\für Fortgeschrittene \\Versuch 2.3}

\date{\today}


%ENDE TITELSEITE

\begin{document}
%ANFANG DOKUMENT

%Titelseite einfügen
\maketitle

%Inhaltsverzeichnis einfügen
\tableofcontents

%ANFANG INHALT

\chapter{Einleitung}

\chapter{Grundlagen}

\section{Kühlverfahren zum Erreichen der Suprafluidität}


\section{Thermometrie bei tiefen Temperaturen}


\section{Paramagnetismus}

\section{Eigenschaften von $^3$He und $^4$He}


\begin{figure}[h] 
  \centering
     \includegraphics[width=0.4\textwidth]{img/aufbau.png}
  \caption{Spezifische Wärmekapazität von $^4$He über der Temperatur. \cite{wiki}}  
  \label{fig:Bild1}
\end{figure}


\begin{center}
  \begin{tabular}{|p{5cm}|p{3cm}|p{3cm}|}
    \hline
    & $^3$He & $^4$He \\ \hline
    Teilchenart & Fermion & Boson  \\ \hline
    Spin & 1/2 & 1  \\ \hline
    Lambdapunkt & 0,0025 K & 2,1768 K  \\ \hline
    Siedepunkt bei 1 bar & 3,19 K & 4,21 K  \\ \hline
	\end{tabular}
\end{center}

\chapter{Aufbau und Durchführung}

\section{Aufnahme der Abkühlkurve}

\section{Aufnahme der Aufwärmkurve}

\chapter{Auswertung}

\section{Druckkorrektur}


\section{Überprüfung des Curie- und Curie-Weiss-Gesetzes}
\begin{figure}[H]
\begin{tikzpicture}
\begin{axis}[
  default,
  title={Kalibration des sekundären Thermometers},
  xlabel=$T$ in $K$,
  ylabel=$U_{ind}$ in $mV$,
  ymin=0,
  ymax=550,
  height=0.5\linewidth
]
\addplot[
  red, only marks, mark=+, mark size=1pt
%, error bars/.cd, y dir=both, y explicit, x dir=both, x fixed relative=0.005
] table[x expr=\coordindex+1, y index=0] {data/air.dat};
\addlegendentry{Messpunkte}
%\addplot[teal, mark=x, mark size=0pt, samples=40, domain=1.4:4.4] {-151.706+753.243/x};
%\addlegendentry{Curie-Fit}
%\addplot[orange, mark=x, mark size=0pt, samples=40, domain=1.4:4.4] {-155.706+769.835/(0.0247+x)};
%\addlegendentry{Curie-Weiss-Fit}
\end{axis}
\end{tikzpicture}
    \caption{}
\end{figure}

\section{Verlauf der Aufwärmung}

\section{Wärmefluss in das Helium}

\chapter{Fazit}
%ENDE INHALT
\cleardoublepage{}
% Eintrag fürs Inhaltsverzeichnis
\newpage
\begin{thebibliography}{100}
  \bibitem{anleitung} Versuchsanleitung zum Versuch Alpha-Spektroskopie, heruntergeladen am 28.06.2015 von der Homepage der TU Darmstadt
  \bibitem{wiki} Grafik aus dem Artikel ---, der freien Enzyklopädie am 21.6.2015 \url{}
  
\end{thebibliography}
\end{document}

%%% Local Variables:
%%% mode: latex
%%% TeX-master: t
%%% End:
